% Options for packages loaded elsewhere
\PassOptionsToPackage{unicode}{hyperref}
\PassOptionsToPackage{hyphens}{url}
%
\documentclass[
]{article}
\usepackage{lmodern}
\usepackage{amssymb,amsmath}
\usepackage{ifxetex,ifluatex}
\ifnum 0\ifxetex 1\fi\ifluatex 1\fi=0 % if pdftex
  \usepackage[T1]{fontenc}
  \usepackage[utf8]{inputenc}
  \usepackage{textcomp} % provide euro and other symbols
\else % if luatex or xetex
  \usepackage{unicode-math}
  \defaultfontfeatures{Scale=MatchLowercase}
  \defaultfontfeatures[\rmfamily]{Ligatures=TeX,Scale=1}
\fi
% Use upquote if available, for straight quotes in verbatim environments
\IfFileExists{upquote.sty}{\usepackage{upquote}}{}
\IfFileExists{microtype.sty}{% use microtype if available
  \usepackage[]{microtype}
  \UseMicrotypeSet[protrusion]{basicmath} % disable protrusion for tt fonts
}{}
\makeatletter
\@ifundefined{KOMAClassName}{% if non-KOMA class
  \IfFileExists{parskip.sty}{%
    \usepackage{parskip}
  }{% else
    \setlength{\parindent}{0pt}
    \setlength{\parskip}{6pt plus 2pt minus 1pt}}
}{% if KOMA class
  \KOMAoptions{parskip=half}}
\makeatother
\usepackage{xcolor}
\IfFileExists{xurl.sty}{\usepackage{xurl}}{} % add URL line breaks if available
\IfFileExists{bookmark.sty}{\usepackage{bookmark}}{\usepackage{hyperref}}
\hypersetup{
  pdftitle={Scenario analysis for healthcare needs for COVID-19 in Georgia},
  pdfauthor={Andreas Handel (ahandel@uga.edu), John M. Drake (jdrake@uga.edu)},
  hidelinks,
  pdfcreator={LaTeX via pandoc}}
\urlstyle{same} % disable monospaced font for URLs
\usepackage[margin=1in]{geometry}
\usepackage{graphicx,grffile}
\makeatletter
\def\maxwidth{\ifdim\Gin@nat@width>\linewidth\linewidth\else\Gin@nat@width\fi}
\def\maxheight{\ifdim\Gin@nat@height>\textheight\textheight\else\Gin@nat@height\fi}
\makeatother
% Scale images if necessary, so that they will not overflow the page
% margins by default, and it is still possible to overwrite the defaults
% using explicit options in \includegraphics[width, height, ...]{}
\setkeys{Gin}{width=\maxwidth,height=\maxheight,keepaspectratio}
% Set default figure placement to htbp
\makeatletter
\def\fps@figure{htbp}
\makeatother
\setlength{\emergencystretch}{3em} % prevent overfull lines
\providecommand{\tightlist}{%
  \setlength{\itemsep}{0pt}\setlength{\parskip}{0pt}}
\setcounter{secnumdepth}{-\maxdimen} % remove section numbering
\usepackage{amsmath}

\title{Scenario analysis for healthcare needs for COVID-19 in Georgia}
\author{Andreas Handel
(\href{mailto:ahandel@uga.edu}{\nolinkurl{ahandel@uga.edu}}), John M.
Drake (\href{mailto:jdrake@uga.edu}{\nolinkurl{jdrake@uga.edu}})}
\date{March 23, 2020}

\begin{document}
\maketitle

\hypertarget{introduction}{%
\subsection{Introduction}\label{introduction}}

The following produces predictions for health care needs in GA for
COVID-19. The basis for this is a transmission simulation model, which
is described here:
\url{http://2019-coronavirus-tracker.com/stochastic-GA.html}

The transmission simulation model is run for different intervention
scenarios. Output from this model is further processed here to produce
estimates for health care needs in GA.

\hypertarget{assumptions}{%
\subsection{Assumptions}\label{assumptions}}

Below are assumptions with high and low values for different quantities
that go into the model predictions.

\hypertarget{age-structure-for-ga}{%
\subsubsection{Age structure for GA}\label{age-structure-for-ga}}

We use the data below for population composition of GA. This age
stratification is used below for risk calculations.

\begin{itemize}
\tightlist
\item
  Total population: 10617423
\item
  Under 18y: 2526947
\item
  Between 18-60y: 7072541
\item
  Between 60-70y: 1017935
\item
  Above 70y: 784892
\end{itemize}

\hypertarget{hospitalization-risk}{%
\subsubsection{Hospitalization risk}\label{hospitalization-risk}}

We assume that \textbf{among those that are cases (i.e.~infected and
tested positive)}, risk of hospitalization is as follows. This is based
on (Wu and McGoogan 2020; Ferguson and al. 2020; Remuzzi and Remuzzi
2020). The first number is a low estimate, the second number a high end
estimate. All values are percent.

\begin{itemize}
\tightlist
\item
  Under 18y: 0.10\%/1.00\%
\item
  Between 18-60y: 1.00\%/10.00\%
\item
  Between 60-70y: 10.00\%/25.00\%
\item
  Above 70y: 20.00\%/50.00\%
\end{itemize}

\hypertarget{critical-care-risk}{%
\subsubsection{Critical Care Risk}\label{critical-care-risk}}

We assume that \textbf{among those that are hospitalized}, risk of
critical care need is as follows. This is based on (Wu and McGoogan
2020; Ferguson and al. 2020; Remuzzi and Remuzzi 2020). The first number
is a low estimate, the second number a high end estimate:

\begin{itemize}
\tightlist
\item
  Under 18y: 1.00\%/10.00\%
\item
  Between 18-60y: 5.00\%/15.00\%
\item
  Between 60-70y: 20.00\%/40.00\%
\item
  Above 70y: 40.00\%/70.00\%
\end{itemize}

\hypertarget{risk-of-death}{%
\subsubsection{Risk of death}\label{risk-of-death}}

We assume that \textbf{among cases}, risk of death is as follows. This
is based on (Verity et al. 2020). The first number is a low estimate,
the second number a high end estimate:

\begin{itemize}
\tightlist
\item
  Under 18y: 0.00\%/0.01\%
\item
  Between 18-60y: 0.10\%/1.00\%
\item
  Between 60-70y: 2.00\%/8.00\%
\item
  Above 70y: 5.00\%/20.00\%
\end{itemize}

\hypertarget{length-of-hospital-stay}{%
\subsubsection{Length of hospital stay}\label{length-of-hospital-stay}}

We assume that a hospital stay is between 10 (low) and 20 (high) days
and independent of age. This is based on (Guan et al. 2020; Tindale et
al. 2020; Sanche et al. 2020).

\hypertarget{model-results}{%
\subsection{Model results}\label{model-results}}

Figures and Tables with predictions for different infection spread
scenarios. See
\href{http://2019-coronavirus-tracker.com/stochastic-GA.html}{here} for
more details on each scenario.

Results show predicted hospitalizations and critical care needs. For
each variable, a low and high prediction is produced. Each variable is
also reported as new individuals entering a given state
(hospitalized/critical/etc.) at any given day and the total number of
individuals in that state at any given date.

\hypertarget{result-figures}{%
\subsubsection{Result Figures}\label{result-figures}}

\includegraphics{healthcare-model-georgia_files/figure-latex/figure1-1.pdf}

\hypertarget{result-tables}{%
\subsubsection{Result Tables}\label{result-tables}}

\begin{verbatim}
## PhantomJS not found. You can install it with webshot::install_phantomjs(). If it is installed, please make sure the phantomjs executable can be found via the PATH variable.
\end{verbatim}

\includegraphics[width=16.18in,height=14.89in,keepaspectratio]{healthcare-model-georgia_files/figure-latex/table1-1.png}

\hypertarget{comments}{%
\subsection{Comments}\label{comments}}

\begin{itemize}
\tightlist
\item
  See (Li et al. 2020) for a similar analysis
\item
  See also propublica:
  \url{https://projects.propublica.org/graphics/covid-hospitals}
\end{itemize}

\hypertarget{references}{%
\subsection*{References}\label{references}}
\addcontentsline{toc}{subsection}{References}

\hypertarget{refs}{}
\leavevmode\hypertarget{ref-ferguson20}{}%
Ferguson, Neil, and et al. 2020. ``Report 9: Impact of
Non-Pharmaceutical Interventions (Npis) to Reduce Covid-19 Mortality and
Healthcare Demand.''
\url{https://www.imperial.ac.uk/mrc-global-infectious-disease-analysis/news--wuhan-coronavirus/}.

\leavevmode\hypertarget{ref-guan2020}{}%
Guan, Wei-jie, Zheng-yi Ni, Yu Hu, Wen-hua Liang, Chun-quan Ou,
Jian-xing He, Lei Liu, et al. 2020. ``Clinical Characteristics of
Coronavirus Disease 2019 in China.'' \emph{New England Journal of
Medicine} 0 (0): null. \url{https://doi.org/10.1056/NEJMoa2002032}.

\leavevmode\hypertarget{ref-li2020}{}%
Li, Ruoran, Caitlin Rivers, Qi Tan, Megan B. Murray, Eric Toner, and
Marc Lipsitch. 2020. ``The Demand for Inpatient and ICU Beds for
COVID-19 in the US: Lessons from Chinese Cities.'' \emph{medRxiv},
March, 2020.03.09.20033241.
\url{https://doi.org/10.1101/2020.03.09.20033241}.

\leavevmode\hypertarget{ref-remuzzi2020}{}%
Remuzzi, Andrea, and Giuseppe Remuzzi. 2020. ``COVID-19 and Italy: What
Next?'' \emph{The Lancet} 0 (0).
\url{https://doi.org/10.1016/S0140-6736(20)30627-9}.

\leavevmode\hypertarget{ref-sanche2020}{}%
Sanche, Steven, Yen Ting Lin, Chonggang Xu, Ethan Romero-Severson, Nick
Hengartner, and Ruian Ke. 2020. ``The Novel Coronavirus, 2019-nCoV, Is
Highly Contagious and More Infectious Than Initially Estimated.''
\emph{medRxiv}, February, 2020.02.07.20021154.
\url{https://doi.org/10.1101/2020.02.07.20021154}.

\leavevmode\hypertarget{ref-tindale2020}{}%
Tindale, Lauren, Michelle Coombe, Jessica E. Stockdale, Emma Garlock,
Wing Yin Venus Lau, Manu Saraswat, Yen-Hsiang Brian Lee, et al. 2020.
``Transmission Interval Estimates Suggest Pre-Symptomatic Spread of
COVID-19.'' \emph{medRxiv}, March, 2020.03.03.20029983.
\url{https://doi.org/10.1101/2020.03.03.20029983}.

\leavevmode\hypertarget{ref-verity2020}{}%
Verity, Robert, Lucy C. Okell, Ilaria Dorigatti, Peter Winskill, Charles
Whittaker, Natsuko Imai, Gina Cuomo-Dannenburg, et al. 2020. ``Estimates
of the Severity of COVID-19 Disease.'' \emph{medRxiv}, March,
2020.03.09.20033357. \url{https://doi.org/10.1101/2020.03.09.20033357}.

\leavevmode\hypertarget{ref-wu2020}{}%
Wu, Zunyou, and Jennifer M. McGoogan. 2020. ``Characteristics of and
Important Lessons from the Coronavirus Disease 2019 (COVID-19) Outbreak
in China: Summary of a Report of 72 314 Cases from the Chinese Center
for Disease Control and Prevention.'' \emph{JAMA}, February.
\url{https://doi.org/10.1001/jama.2020.2648}.

\end{document}
